\section{Introduction}
\label{sec:intro}

The Internet of Things (IoT) provides tremendous opportunities for
enhanced quality of life; however, the realization of these opportunities
requires both the technical ability to deliver them and the economic
incentives to provide them.
Clean water is a benefit to all, and generally is taken for granted
by the citizenry of the industrialized world.
Yet, \FIXME{talk about challenges and how IoT can make it better.}

BECS Technology, Inc., is a firm that provides water chemistry monitoring
and control equipment to the aquatics market, which includes a number
of (mostly distinct) market segments (e.g., pools, including municipal pools,
commercial pools, water parks, health clubs, animal habitats in zoos,
residential pools; drinking water treatment facilities; waste water
treatment facilities; fountains; building cooling towers; irrigation
systems; and various commercial operations, including washing vegetables,
fracking; etc.).
Its BECSys\texttrademark{} line of controllers measure
a number of aspects of water chemistry (e.g., pH,
oxidation-reduction potential (ORP), free chlorine concentration,
temperature, conductivity, turbidity, alkalinity~\cite{cew18},
etc.) and perform various control operations (e.g., feed chemical)
to ensure that the water chemistry stays within desired parameters.

EZConnect\texttrademark{}
is the security infrastructure BECS has developed to provide
remote access capability to equipment it manufactures.  This remote
access capability satisfies the need for security yet balances
that need with the equivalent need for ease of installation and
maintenance~\cite{ezconnect,ccgss18}.

Here, we describe EZAnalytics\texttrademark{}, the use of the EZConnect infrastructure to enable
cloud-based data logging and analytics on water chemistry data from a
collection of controllers.

\FIXME{Introduce e-commerce angle.
Ease of use is important in e-commerce~\cite{gefen2000}.}
