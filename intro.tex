\section{Introduction}
\label{sec:intro}

The Internet of Things (IoT) provides tremendous opportunities for
enhanced quality of life; however, the realization of these opportunities
requires both the technical ability to deliver them and the economic
incentives to provide them.
Clean water is a benefit to all, and generally is taken for granted
by the citizenry of the industrialized world.
Yet, the maintenance of water chemistry across the aquatics industry
is frequently the responsibility of a large number of small operators,
which limits their ability to aggregate or learn from data that 
represents a broad set of circumstances.

Collecting, aggregating, and learning from data across the industry
is now technically feasible via the internet of things,
and this can be accomplished while
preserving the confidentiality of individual organizations that
contribute their data to the community.

BECS Technology, Inc., is a firm that provides water chemistry monitoring
and control equipment to the aquatics market, which includes a number
of (mostly distinct) market segments (e.g., pools, including municipal pools,
commercial pools, water parks, health clubs, animal habitats in zoos,
residential pools; drinking water treatment facilities; waste water
treatment facilities; fountains; building cooling towers; irrigation
systems; and various commercial operations, including washing vegetables,
fracking; etc.).
Its BECSys\texttrademark{} line of controllers measure
a number of aspects of water chemistry (e.g., pH,
oxidation-reduction potential (ORP), free chlorine concentration,
temperature, conductivity, turbidity, alkalinity~\cite{cew18},
etc.) and perform various control operations (e.g., feed chemical)
to ensure that the water chemistry stays within desired parameters.

EZConnect\texttrademark{}
is the security infrastructure BECS has developed to provide
remote access capability to equipment it manufactures.  This remote
access capability satisfies the need for security yet balances
that need with the equivalent need for ease of installation and
maintenance~\cite{ezconnect,ccgss18}.

Here, we describe EZAnalytics\texttrademark{}, the use of the
EZConnect infrastructure to enable
cloud-based data logging and analytics on water chemistry data from a
collection of controllers.
With a definition of e-commerce that is broad enough to include commercial
benefit due to the transfer of data over networks, 
the EZAnalytics framework supports improved inventory control,
reduced maintenance overhead, safer operation of water chemistry
installations, and ultimately higher quality water.

Yet, none of the benefits described above will be realized if
the system is not straightforward to use.
Gefen and Straub~\cite{gefen2000} provide a cautionary tale
in which we must not only ensure that the system is easy to use,
we must also be concerned about the \emph{perceived} ease of use.
Individuals (and organizations) will not alter their standard
practices unless the perceived usefulness of adopting a technological
change outweighs the costs of adoption (strongly impacted by
the perceived ease of learning and ease of use)~\cite{Davis89}.
From their inception, both EZConnect and EZAnalytics have been
designed to be well matched to their users' experience
(i.e., they are designed for those whose knowledge is in
water chemistry, not necessarily internet connectivity or data analysis).

The paper is organized as follows. Section~\ref{sec:background}
provides background information about the water chemistry controllers
and how they are connected to the network.  This is followed by a
description of related work in the area of e-commerce and IoT.
Section~\ref{sec:cloud} tells how controller data are delivered to
the cloud, and Section~\ref{sec:analytics} describes the benefits
that can accrue from analyzing the data.
Section~\ref{sec:conclude} concludes and describes future work.
