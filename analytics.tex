\section{Data Analytics}
\label{sec:analytics}

While the summary data presentation is valuable to users, particularly for
understanding the current state of the water chemistry that is immediately
their responsibility, the true value of the data in the cloud is what it
can potentially tell us about water treatment generally, and the impact
that it can have commercially.
By combining data from a number of controllers, we can learn more than
it is possible to discern from any one example.

We are aggregating controller data in two ways.  First, data that are
from controllers all owned by an individual organization can be readily
combined and mined for the benefit of that organization.  They own
both the equipment and the data. Second, for those organizations that
give permission, anonymized data sets can be made available for mining
(e.g., see~\cite{horey2007,zhong2009k}), providing for a potentially
much larger data set.

There are a number of things we hope to learn from these data.
\begin{itemize}
\item The effectiveness of control decisions.  How well do different
approaches work at maintaining water chemistry control?  What factors
are important?
\item How to recommend better control approaches.  How well will control
approaches that are effective in one body of water transfer to another?
\item How to minimize chemical usage.  The data logs not only contain
sensor readings of water chemistry, but also log chemical feed rates
and supplies.  Can we make water chemistry control more cost effective,
by decreasing the consumables needed?
\item How to predict problems. Alarm conditions (e.g., due to water
chemistry being out of balance) require fast and expensive human response.
Both public safety and maintenance effort required will be improved
if we can predict problems prior to them reaching the level of an alarm.
\end{itemize}

Modern machine learning techniques are well suited to addressing the
questions we pose above.
While the data analysis has not yet happened, as the system is just
now being deployed in the field, we are very optimistic that new insights
will come from the data aggregation described above.

The first example of commercial impact is on inventory monitoring
and resupply. It is frequently the case that chemical inventories
are monitored by the controllers, while can straightforwardly enable
automated re-ordering of chemical as well as scheduling of delivery.
Not only does this reduce the manual workload associated with
inventory control, it also reduces cases of inadvertent supply outages
which can result in erroneous water chemistry.

The second example is a significant reduction in maintenance costs
for operators.  By alerting operators to potential
issues with the water chemistry before it reaches alarm levels, 
the organization can be much more flexible in scheduling service
personnel. E.g., an operator can investigate the issue remotely,
querying the controller for relevant information; the need for
a dispatch to the physical site need not be an emergency, but can
instead by folded into a more convenient schedule; and when the
issue can be resolved in time, an alarm condition can be averted.

The above to examples are cases where the data of interest is specific
to the bodies of water (and associated controllers) that are the
responsibility of an individual organization.  Tossing our net
even wider, by aggregating data from disparate organizations, it
is possible to gain even more benefit.

As an example of benefits gained by data aggregation, improved
control techniques can be more widely disseminated and used.
In the context of recreational water, a million gallon competition
pool doesn't react in the same way as a lazy river, or a shallow
splash pool designed for the very young. Data aggregation across
organizations allows for operators to see the effects of control
techniques on each of these bodies of water separately, yet with
sufficient examples of each type that general conclusions can
reasonably be drawn.

These control techniques might be simple things, such as parameter
settings in the existing control algorithms, or they might be
more involved, such as more sophisticated control algorithms
(i.e., under what conditions time proportional control~\cite{McP13}
is beneficial). The overall goal, however, is the same in each
case. We want to improve water quality control by improved techniques
driven by data analysis across multiple sites and organizations.

The e-commerce implications of this can be significant, with the
primary benefits being reduced chemical usage and reduced maintenance
costs (due to fewer issues that need to be addressed).

The above can all be accomplished while preserving the confidentiality
of the data associated with specific sites.  There are a number
of robust techniques now available for sharing aggregate data
while maintaining originators' anonymity~\cite{horey2007,zhong2009k}.
