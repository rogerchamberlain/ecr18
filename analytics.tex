\section{Data Analytics}
\label{sec:analytics}

While the summary data presentation is valuable to users, particularly for
understanding the current state of the water chemistry that is immediately
their responsibility, the true value of the data in the cloud is what it
can potentially tell us about water treatment generally.
By combining data from a number of controllers, we can learn more than
it is possible to discern from any one example.

We are aggregating controller data in two ways.  First, data that are
from controllers all owned by an individual organization can be readily
combined and mined for the benefit of that organization.  They own
both the equipment and the data. Second, for those organizations that
give permission, anonymized data sets can be made available for mining
(e.g., see~\cite{horey2007,zhong2009k}), providing for a potentially
much larger data set.

There are a number of things we hope to learn from these data.
\begin{itemize}
\item The effectiveness of control decisions.  How well do different
approaches work at maintaining water chemistry control?  What factors
are important?
\item How to recommend better control approaches.  How well will control
approaches that are effective in one body of water transfer to another?
\item How to minimize chemical usage.  The data logs not only contain
sensor readings of water chemistry, but also log chemical feed rates
and supplies.  Can we make water chemistry control more cost effective,
by decreasing the consumables needed?
\item How to predict problems. Alarm conditions (e.g., due to water
chemistry being out of balance) require fast and expensive human response.
Both public safety and maintenance effort required will be improved
if we can predict problems prior to them reaching the level of an alarm.
\end{itemize}

Modern machine learning techniques are well suited to addressing the
questions we pose above.
While the data analysis has not yet happened, as the system is just
now being deployed in the field, we are very optimistic that new insights
will come from the data aggregation described above.
